\section{Motivation}
Automatic text summarization is an important task in natural language processing. With the ever-growing amount of textual material emerging, the ability to efficiently and accurately generate summaries becomes crucial. Summaries can allow us to quickly index documents and make selections in a much shorter period of time. It is also less prone to personal bias when performed by a machine. Therefore, in this project, we will explore deep learning methods to hopefully provide a general approach to this problem.

\section{Problem Statement}

We want to ....

\section{Methods}
\begin{compactitem}
\item Use deep learning. 
\end{compactitem}

\todo{How does your project compare to NLP and deep learning published research?}

\todo{What are the different steps you will take to ensure success of the project? What are the smaller segments of which the bigger task is composed? And how will you conduct each small task?
How does the work bread down and what each member of the team be contributing?}

\section{Evaluation}
\begin{compactitem}
\item Evaluation1
\end{compactitem}

\section{Tasks and Assignment}

\begin{compactitem}
\item Task 1 (Fan, Meng)
\item Task 2 (Yin, Meng)
\item Task 3 (Yin)
\item Task 4 (Fan)
\end{compactitem}


\section{Timeline}
\begin{compactitem}
\item Week 1: do task 1
\item Week 2: do task 2
\item Week 3: do task 3
\item Week 4: do task 3 and task 4
\end{compactitem}

\bibliography{acl2018}
\bibliographystyle{acl_natbib}
